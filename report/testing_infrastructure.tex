%---------------------------------------------------------------------------------
\chapter{Code testing}
\label{chap:code-testing}
%---------------------------------------------------------------------------------
In the progress of constructing the software, a unit testing infrastructure was put in place. The purpose of the unit testing is to create a robust and sustainable software. Written codes are tested to make sure it runs as intended and returns expected results. If code is tested while writing, it would be easier to fix bugs in the future as previously tested code would be correct and most probably free of bugs. Code coverage is used to define the percentage of codes covered in the unit testing process. It is usually aim to achieve a 100\% code coverage. However, a 100\% code coverage does not necessarily mean that the code is correct or free of errors. Nevertheless, it provides some confidence that the code is implemented correctly.

Every method in this numerical-solver software is tested. The numerical solution of each method is tested to give the same solution as a manually calculated solution. The methods are mostly tested against the example model \ref{eqn:example_model}.

The methods are classified into three classes: one-step methods, predictor-corrector methods and adaptive methods. The initialisations of each class are tested to ensure that variables are initialised correctly and input type satisfies the requirements. For example, 

\begin{lstlisting}[language=Python]
def test__init__(self):

    def func(x, y):
        return [-y[0]]
    x_min = 0
    x_max = 1
    initial_value = [1]
    mesh_points = 10

    problem = solver.OneStepMethods(
        func, x_min, x_max, initial_value, mesh_points)

    # Test initialisation
    self.assertEqual(problem.x_min, 0)
    self.assertEqual(problem.mesh_points, 10)

    # Test raised error for callable function
    with self.assertRaises(TypeError):
        solver.OneStepMethods(
            x_min, x_min, x_max, initial_value, mesh_points)

    # Test raised error if initial_value not list
    with self.assertRaises(TypeError):
        solver.OneStepMethods(
            func, x_min, x_max, 1, mesh_points)
\end{lstlisting}
. A simple model with analytical solution is first initialised. Required inputs were check to make sure the problem is set up properly, in line 14 and 15 of the code snippet above. To ensure that the inputs to the function are of the desired data type. 

% Add description to test of error raised